\usepackage[utf8]{inputenc}
\usepackage[T1]{fontenc}
\usepackage{textcomp}
\usepackage[portuguese]{babel}

\usepackage{graphicx}
\usepackage{float}
\usepackage{xcolor}

\usepackage{amsmath, amsfonts, mathtools, amsthm, amssymb, lipsum, pgfplots, siunitx}
\usepackage{mathrsfs}
\usepackage{cancel}

\pgfplotsset{compat=1.18}
\sisetup{inter-unit-product =$\cdot$}

\usepackage
[
   a4paper,% other options: a3paper, a5paper, etc
   vmargin=2cm,
   hmargin=2cm
]
{geometry}

\usepackage{verse}
\def\signed #1{{\leavevmode\unskip\nobreak\hfil\penalty50\hskip2em
\hbox{}\nobreak\hfil(#1)%
\parfillskip=0pt \finalhyphendemerits=0 \endgraf}}

\newsavebox\mybox
\newenvironment{biblequote}[1]
{\savebox\mybox{#1}\begin{quote}}
{\signed{\usebox\mybox}\end{quote}}

\def\verse#1#2 {\raisebox{.5ex}{\scriptsize#1}\textit{#2 }}
\newcommand\tverse[2]{\vin \verse{#1}{#2}}

% horizontal rule
\newcommand\hr{
\noindent\rule[0.5ex]{\linewidth}{0.5pt}
}

\usepackage{hyperref}
\hypersetup{
    colorlinks,
    linkcolor={black},
    citecolor={black},
    urlcolor={blue!80!black}
}

\usepackage{tikz}
\usepackage{tikz-cd}
\usepackage{pict2e}

\makeatletter
\newcommand{\crossout}[1]{%
   \begingroup
   \settowidth{\dimen@}{#1}%
   \setlength{\unitlength}{0.05\dimen@}%
   \settoheight{\dimen@}{#1}%
   \count@=\dimen@
   \divide\count@ by \unitlength
   \begin{picture}(0,0)
   \put(0,0){\line(20,\count@){20}}
   \put(0,\count@){\line(20,-\count@){20}}
   \end{picture}%
   #1%
   \endgroup
}
\makeatother

% theorems
\usepackage{thmtools}
\usepackage[framemethod=TikZ]{mdframed}
\mdfsetup{skipabove=1em,skipbelow=0em, innertopmargin=5pt, innerbottommargin=6pt}


\theoremstyle{definition}

\makeatletter


\@ifclasswith{report}{nocolor}{
    \declaretheoremstyle[headfont=\bfseries\sffamily, bodyfont=\normalfont]{thmstyle}
    \declaretheoremstyle[headfont=\bfseries\sffamily, bodyfont=\normalfont]{corostyle}
    \declaretheoremstyle[headfont=\bfseries\sffamily, bodyfont=\normalfont]{defstyle}
    \declaretheoremstyle[headfont=\bfseries\sffamily, bodyfont=\normalfont]{egstyle}
    \declaretheoremstyle[headfont=\bfseries\sffamily, bodyfont=\normalfont]{exstyle}
    \declaretheoremstyle[headfont=\bfseries\sffamily, bodyfont=\normalfont]{propstyle}
    \declaretheoremstyle[headfont=\bfseries\sffamily, bodyfont=\normalfont]{lemstyle}
    \declaretheoremstyle[headfont=\bfseries\sffamily, bodyfont=\normalfont]{proofstyle}
    \declaretheoremstyle[headfont=\bfseries\sffamily, bodyfont=\normalfont]{explstyle}
    \declaretheoremstyle[headfont=\bfseries\sffamily, bodyfont=\normalfont]{remarkstyle}
    \declaretheoremstyle[headfont=\bfseries\sffamily, bodyfont=\normalfont]{notecolor}
    \declaretheoremstyle[headfont=\bfseries\sffamily, bodyfont=\normalfont]{solstyle}
}{
\definecolor{ThmColor}{HTML}{1F75FE}
\definecolor{CoroColor}{HTML}{008B8B}
\definecolor{DefColor}{HTML}{DC143C}
\definecolor{EgColor}{HTML}{32CD32}
\definecolor{ExColor}{HTML}{009F6B}
\definecolor{PropColor}{HTML}{32174D}
\definecolor{LemColor}{HTML}{7851A9}
\definecolor{ProofColor}{HTML}{717171}
\definecolor{ExplColor}{HTML}{F4A460}
\definecolor{RemarkColor}{HTML}{FF7F50}
\definecolor{NoteColor}{HTML}{d68a59}
\definecolor{SolColor}{HTML}{EEED09}

    \declaretheoremstyle[
        headfont=\bfseries\sffamily\color{DefColor!70!black}, bodyfont=\normalfont,
        headpunct={},
        postheadspace=\newline,
        mdframed={
            linewidth=2pt,
            rightline=false, topline=false, bottomline=false,
            linecolor=DefColor, backgroundcolor=DefColor!5,
        }
    ]{defstyle}
    
    \declaretheoremstyle[
        headfont=\bfseries\sffamily\color{EgColor!70!black}, bodyfont=\normalfont,
        headpunct={},
        postheadspace=\newline,
        mdframed={
            linewidth=2pt,
            rightline=false, topline=false, bottomline=false,
            linecolor=EgColor, backgroundcolor=EgColor!5,
        }
    ]{egstyle}

    \declaretheoremstyle[
        headfont=\bfseries\sffamily\color{ExColor!70!black}, bodyfont=\normalfont,
        headpunct={},
        postheadspace=\newline,
        mdframed={
            linewidth=2pt,
            rightline=false, topline=false, bottomline=false,
            linecolor=ExColor, backgroundcolor=ExColor!5,
        }
    ]{exstyle}

    \declaretheoremstyle[
        headfont=\bfseries\sffamily\color{PropColor!70!black}, bodyfont=\normalfont,
        headpunct={},
        postheadspace=\newline,
        mdframed={
            linewidth=2pt,
            rightline=false, topline=false, bottomline=false,
            linecolor=PropColor, backgroundcolor=PropColor!5,
        }
    ]{propstyle}

    \declaretheoremstyle[
        headfont=\bfseries\sffamily\color{ThmColor!70!black}, bodyfont=\normalfont,
        headpunct={},
        postheadspace=\newline,
        mdframed={
            linewidth=2pt,
            rightline=false, topline=false, bottomline=false,
            linecolor=ThmColor, backgroundcolor=ThmColor!5,
        }
    ]{thmstyle}

    \declaretheoremstyle[
        headfont=\bfseries\sffamily\color{LemColor!70!black}, bodyfont=\normalfont,
        headpunct={},
        postheadspace=\newline,
        mdframed={
            linewidth=2pt,
            rightline=false, topline=false, bottomline=false,
            linecolor=LemColor, backgroundcolor=LemColor!5,
        }
    ]{lemstyle}

    \declaretheoremstyle[
        headfont=\bfseries\sffamily\color{CoroColor!70!black}, bodyfont=\normalfont,
        headpunct={},
        postheadspace=\newline,
        mdframed={
            linewidth=2pt,
            rightline=false, topline=false, bottomline=false,
            linecolor=CoroColor, backgroundcolor=CoroColor!5,
        }
    ]{corostyle}

    \declaretheoremstyle[
        headfont=\bfseries\sffamily\color{ExplColor!70!black}, bodyfont=\normalfont,
        headpunct={},
        postheadspace=\newline,
        numbered=no,
        mdframed={
            linewidth=2pt,
            rightline=false, topline=false, bottomline=false,
            linecolor=ExplColor, backgroundcolor=ExplColor!5,
        }
    ]{explstyle}

    \declaretheoremstyle[
        headfont=\bfseries\sffamily\color{SolColor!70!black}, bodyfont=\normalfont,
        headpunct={},
        postheadspace=\newline,
        numbered=no,
        mdframed={
            linewidth=2pt,
            rightline=false, topline=false, bottomline=false,
            linecolor=SolColor, backgroundcolor=SolColor!5,
        }
    ]{solstyle}

    \declaretheoremstyle[
        headfont=\bfseries\sffamily\color{RemarkColor!70!black}, bodyfont=\normalfont,
        headpunct={},
        postheadspace=\newline,
        mdframed={
            linewidth=2pt,
            rightline=false, topline=false, bottomline=false,
            linecolor=RemarkColor, backgroundcolor=RemarkColor!5,
        }
    ]{remarkstyle}

    \declaretheoremstyle[
        headfont=\bfseries\sffamily\color{NoteColor!70!black}, bodyfont=\normalfont,
        headpunct={},
        postheadspace=\newline,
        mdframed={
            linewidth=2pt,
            rightline=false, topline=false, bottomline=false,
            linecolor=NoteColor, backgroundcolor=NoteColor!5,
        }
    ]{notestyle}

    \declaretheoremstyle[
        headfont=\it\bfseries\sffamily\color{ProofColor!70!black}, bodyfont=\normalfont,
        headpunct={},
        postheadspace=\newline,
        numbered=no,
        mdframed={
            linewidth=2pt,
            rightline=false, topline=false, bottomline=false,
            linecolor=ProofColor, backgroundcolor=ProofColor!1,
        },
        qed=\qedsymbol
    ]{proofstyle}
}


\declaretheorem[style=defstyle, numberwithin=chapter, name=Definição]{definition}
\declaretheorem[style=egstyle, numberwithin=chapter, name=Exemplo]{eg}
\declaretheorem[style=exstyle, numberwithin=chapter, name=Exercício]{ex}
\declaretheorem[style=propstyle, numberwithin=chapter, name=Proposição]{proposition}
\declaretheorem[style=thmstyle, numberwithin=chapter, name=Teorema]{theorem}
\declaretheorem[style=lemstyle, numberwithin=chapter, name=Lema]{lemma}
\declaretheorem[style=corostyle, numberwithin=chapter, name=Corolário]{corollary}

\@ifclasswith{report}{nocolor}{
    \declaretheorem[style=proofstyle, name=Demonstração]{replacementproof}
    \declaretheorem[style=explstyle, name=Explicação]{explanation}
    \declaretheorem[style=solstyle, name=Solução]{solution}
    \renewenvironment{proof}[1][\proofname]{\begin{replacementproof}}{\end{replacementproof}}
}{
    \declaretheorem[style=proofstyle, name=Demonstração]{replacementproof}
    \renewenvironment{proof}[1][\proofname]{\vspace{-12pt}\begin{replacementproof}}{\end{replacementproof}}

    \declaretheorem[style=explstyle, name=Explicação]{tmpexplanation}
    \newenvironment{explanation}[1][]{\vspace{-12pt}\begin{tmpexplanation}}{\end{tmpexplanation}}

    \declaretheorem[style=solstyle, name=Solução]{tmpsolution}
    \newenvironment{solution}[1][]{\vspace{-12pt}\begin{tmpsolution}}{\end{tmpsolution}}
}

\makeatother

\declaretheoremstyle[
   headpunct={},
]{previousstyle}

\declaretheorem[style=remarkstyle, numberwithin=chapter, name=Observação]{remark}
\declaretheorem[style=notestyle, numbered=no, name=Nota]{note}
\declaretheorem[style=previousstyle, numbered=no, name=Relembrando...]{previouslyseen}


% http://tex.stackexchange.com/questions/22119/how-can-i-change-the-spacing-before-theorems-with-amsthm
% \def\thm@space@setup{%
%   \thm@preskip=\parskip \thm@postskip=0pt
% }


% fancy headers
\usepackage{fancyvrb}
\usepackage{fancyhdr}
\pagestyle{fancy}
\fancyhead[L]{\nouppercase\leftmark}
\fancyhead[RO]{\nouppercase\rightmark}
\setlength\headheight{15pt}


% figure support (https://castel.dev/post/lecture-notes-2)
\usepackage{import}
\usepackage{pdfpages}
\usepackage{transparent}
\usepackage{xcolor}

\newcommand{\incfig}[2][1]{
    \def\svgwidth{#1\columnwidth}
    \import{./figures/}{#2.pdf_tex}
}
\pdfsuppresswarningpagegroup=1

% http://tex.stackexchange.com/questions/76273/multiple-pdfs-with-page-group-included-in-a-single-page-warning
\pdfsuppresswarningpagegroup=1

\author{Felipe Scherer Vicentin}
